\documentclass[a4paper,10pt]{article}
\usepackage[utf8]{inputenc}
\usepackage[frenchb]{babel}
\usepackage{graphicx}
\usepackage{amsthm}
\usepackage{amsmath}
\graphicspath{{png/}}
%opening
\title{Algorithmique avancée : Rapport du projet}
\author{FOURNIER Benoit, THIERRY Constance}

\begin{document}

\maketitle

\begin{figure}[b]
\begin{center}
\includegraphics[scale=1]{Enssat.png}
\end{center}
\end{figure}

\thispagestyle{empty}

\newpage
\null
\thispagestyle{empty}
\newpage

\tableofcontents

\hfill

\listoffigures

\newpage
\null
\thispagestyle{empty}
\newpage

\section{Introduction}

\section{Etude préliminaire}

\subsection{Nombre de cordes distinctes dans un polygone à \emph{n} sommets}

\paragraph{Propriété :}
Appelons \emph{NbCordesDistinctes(n)} le nombre de cordes distinctes que l'on peut dénombrer dans un polygone à \emph{n} sommets.
Nous avons alors :

\begin{equation} 
\begin{array}{r @{=} l}
NbCordesDistinctes(n) \ & \ 2(n-3) + \sum_{i=0}^{n-4} i \\
		      & \ 2(n-3) + \frac{(n-4)(n-3)}{2} \\
		      & \ \frac{n(n-3)}{2}
\end{array} 
\end{equation}


\begin{proof}
Pour \(n = 4\) nous avons : \(NbCordesDistinctes(4)=2 \) la propriété est vérifiée. \\
Admetons la propriété vrais à un rang \emph{n} et démontrons la au rang \emph{n+1}. \\
Nous avons :\\
\[NbCordesDistinctes(n+1) = (NombreDeNouvellesCordes) + NbCordesDistinctes(n)\]
En effet, le résultat du problème au rang \emph{n+1} est égal à celui du rang \emph{n} auquel on ajoute les nouvelles cordes apparues.
Ce nombre de nouvelles cordes est égale à \(n - 2 + 1 \) car lorsque l'on considère un sommet supplémentaire, celui-ci peut ce lier aux n anciens sommets.
On lui retire les deux sommets qui lui sont adjacents, et on rajoute une liaison entre ces deux sommets adjacents.

%%TODO: ajouter un shéma car pas évident

\[
\begin{array}{r @{=} l}
NbCordesDistinctes(n+1) \ & \ n -  2 + 1 + NbCordesDistinctes(n) \\
			  & \ n - 1 + \frac{n(n-3)}{2} \\
			  & \ \frac{{n^2}-n-2}{2} \\
			  & \ \frac{(n+1)(n-2)}{2} \\
\end{array}
\]
Ce qui achève la récurrence.	
\end{proof}


\subsection{Nombre de cordes pour les triangulation d'un polygone à \emph{n} sommets}

\paragraph{Propriété :}  
Appelons \emph{NbCordesTriang(n)} le nombre de cordes présentent dans une triangulation d'un polygone à \emph{n} sommets.
Toutes les triangulations d'un polygone à \emph{n} sommets comportent le même nombre de cordes \emph{NbCordesTriang(n)}.
De plus nous avons :
\begin{equation} 
NbCordesTriang(n) = n-3
\end{equation}

\begin{proof}
Pour \(n = 4\) nous avons : \(NbCordesTriang(4) = 1 \) la propriété est vérifiée. \\
%TODO : ajouter trop beaux shéma
Admetons la propriété vrais à un rang \emph{n} et démontrons la au rang \emph{n+1}. \\
Considerons un polygone à \emph{n+1} sommets, en traçant une corde entre les sommets \emph{i} et \emph{i+2} on génère un sous polygone de taille \emph{n}.
Le nombre de cordes d'une triangulation d'un polygone de taille \emph{n+1} est donc égale au nombre de cordes du sous polygone de taille \emph{n} au quel on ajoute une corde.
Nous avons :\\
\[
\begin{array}{r @{=} l}
NbCordesTriang(n+1) \ & \  1 corde + NbCordesTriang(n) \\
			  & \ 1 + (n-3) \\
			  & \ (n+1) - 3
\end{array}
\]
Ce qui achève la récurrence.
\end{proof}


\section{Essais successifs}

\section{Programmation dynamique}

\subsection{Formule de récurrence}

\subsection{Algorithme}

\subsection{Compléxité spatiale et temporelle}

\subsection{Améliorations possibles}

\section{Algorithme glouton}

\section{Conclusion}


\end{document}
